
\chapter{Approach}
This chapter begins with a presentation of the method used in this thesis, followed by a section explaining the demo application developed as part of the method. The last section contains a presentation of recent articles on the subject. 
\section{Method}
??
In order to compare the two development methods (described in problem formulation), we have tried using them both to 

In order to research the problem formulation of this paper, a practical approach has been used. The two different development methods (as described in problem formulation) have been used in a practical development process. The aim of the development has been to develop an architecture allowing independent web application code and mobile-application code to be written.

Apart from just developing architectures using the different methods, a demo-application was developed. The application had the same function requirements for both development methods.

This also put requirements on the web application, it had to confirm to a certain standard for communication with the mobile application in order for both applications to be able to use it. 

Another important aspect of the method, and a major part of the results found in this paper is the process of researching questions arising during the process. Examples of this can be questions regarding strengths and weaknesses of the method at hand, but also questions of a more basic nature, such as the behavior of function calls between the layers (web and mobile), are they synchronous?

\subsection{Demo application}
The demo application consists two mobile applications developed using the different development methods. The web application which is used in the mobile applicaitons is a simple form with a text field, an image field and a location field.

When running on a computer in a web browser the web application make use of the computer's camera and the web browser's own function to get the computers location.

The two mobile applications make use of the web application described above. To get the image for the web application form the mobile application makes use of the mobile's native camera function or uploads an image from the mobile's memory storage. To get the location the mobile applications makes use of the mobile's native GPS function. 

Application source code can be found at:
\begin{itemize}
  \item Web application: \url{https://github.com/Albin-trialbee/pollux-server}
  \item PhoneGap application: \url{https://github.com/DavidNorrestam/pollux-phonegap}
  \item Native Android mobile application: \url{https://github.com/buren-trialbee/pollux/}
\end{itemize}