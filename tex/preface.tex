\author{
	David Norrestam \\
	{\normalsize \href{mailto:davidnorrestam@gmail.com}{\texttt{davidnorrestam@gmail.com}}}
	\and
	Philip Burenstam Linder \\
    {\normalsize \href{mailto:philip.burenstam.linder@gmail.com}{\texttt{philip.burenstam.linder@gmail.com}}}
}

\title{Hybrid app-development using an existing web application}
%\subtitle{A feasibility study}
\company{Lund University}

%\date{\today}
\date{Month Day, 2015}

\supervisors{Flavius Gruian, \href{mailto:Flavius.Gruian@cs.lth.se}{\texttt{Flavius.gruian@cs.lth.se}}}{Albin Willman, \href{mailto:albin.svensson@trialbee.com}{\texttt{albin.svensson@trialbee.com}}}
%\supervisor{John Deer, \href{mailto:jdeer@company.com}{\texttt{jdeer@company.com}}}
\examiner{Flavius Gruian, \href{mailto:flavius.gruian@cs.lth.se}{\texttt{flavius.gruian@cs.lth.se}}}


\acknowledgements{
We wish to offer our thanks to Trialbee for giving us the idea for this investigation and our supervisor Albin Willman who guided us and helped us develop the web application.

We would also like to take this oppurtunity to thank our examiner, Flavius Gruian at the Department of Computer Science, for all the advices and feedback he has provided throughout the process of writing this thesis.

\theabstract{
Mobile applications are today an important way for companies to reach their customers. Developing a mobile application could require a lot of resources, especially if the application is to be made from scratch. However, an alternative for a company with an existing web application, is making use of the logic in the existing web application, to create a hybrid mobile application.

In this thesis, two different approaches to developing an Android application were evaluated. The application to be developed consisted of classes for communication with native Android functions, and an encapsulated (existing) web application, making use of the functions provided by the aforementioned classes to extend its functionality. In one of the approaches, Android Framework was used for developing the hybrid application, and in the other approach, PhoneGap framework was used.

For evaluation, we recorded the development effort required in each of the two approaches, for a quantitative comparison, and also examined the structure of the developed applications, for a qualitative comparison. Development effort was estimated by measuring logical lines of code (LLoC) of the resulting application. 

The results show that a lower development effort is required when developing using PhoneGap framework, than in the Android framework. However, we noticed that developing using Android framework provides more control over the application life cycle, and can thus be a preferable option when a more advanced application needs to be developed.
}

\keywords{android, hybrid, phonegap, mobile application, web application}