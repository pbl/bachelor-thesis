\chapter{Discussion} \label{ch:discussion}
The problem formulation was:
``What are the advantages and disadvantages for two different development methods for encapsulating an existing web application in a mobile application to utilize native functions? An example of such a native function would be the mobile's accelerometer. The advantages and disadvantages will be evaluated by looking at development effort, and how the code can be structured in the mobile and web application.’’

Developing in the PhoneGap frame requires less development effort than in the Android framework, a reason for this might be that calls to native functions are handled on a higher abstraction. The downside of accessing the native functions on a higher abstraction is that you have less freedom in terms of customizing the behavior. That doesn’t make any difference for a simple mobile application such as the one developed but for a bigger and more advanced application it might be a problem. 

Communication between web and mobile application was made at the cost of a security flaw in the PhoneGap framework. The developed mobile application encapsulated the web application by use of an inline frame. To be able to encapsulate the web application in an inline frame the web applications X-Frame-Options response header was removed which makes it possible to use the web application for ClickJacking, see~\ref{subsec:inline-frame}. 

In the Android framework there is a closer relation to the Android System/Hardware, however this comes with a requirement of lots of code for basic functionality. This closer relation could for example mean a better control of the application lifecycle. Basic knowledge of Android is needed, if the developer has no prior knowledge of Android this means a starting cost. 

Once the architectural structure was in place in the Android framework it was easy to add new functionality to the application. The structure is very modular separating creation of an intent, handling the result of an intent and the communication with the web application layer. To set up this behavior relatively many classes was needed since Java is an Object oriented language. The number of classes was one of the reasons that the developed mobile application in the Android framework had more than double the amount of logical lines of code. 


\iffalse
\section{Developing in Android}
\subsection{Getting started}
\subsection{Debugging}
Debugging when using the Android Studio IDE is a fairly simple process. For developers familiar with Java-debugging, the process is as good as similar, apart from the fact that the code is run on an emulator, or real device, rather than on directly by the JVM. Complications do occur when the application contains a WebView or other browser and the aim is the debug both the web and java code. However, with the help of the chrome inspect tool, the web code is easily debugged, and debugging of communication in between Java and JavaScript is easiest made by a combination of this tool and the built-in debugger. 
\section{Developing in PhoneGap}
\subsection{Getting started}
Getting familliar with PhoneGap is quite confusing. This resulting in that it took around two days to set up a working development environment that felt comfortable. There were a few reasons for this. You need to install a lot dependencies such as nodejs, a server-side runtime environment, and Apache Cordova. Apache Cordova can be described as the engine of PhoneGap. There is also several different ways to develop the mobile application. You can compile and build the app for testing yourself or use a cloud service like https://build.phonegap.com/ to compile your code for you. The relationship between Apache Cordova and PhoneGap is also initially confusing. It was also very difficult to find a good “Getting started” guide.

\subsection{Adding and debbuging native functionality}
The documenation is clear and feels complete. If you require extra help the PhoneGap community is very active and there is lots of help to get from blog posts and Q\&A sites such as stackoverflow.com. This make’s it very simple to make use of the mobile’s native functions. 

When using mobile native function a plugin need to be added to the project. The plugin can be supported by PhoneGap, a third party plugin or a plugin you write yourself. A PhoneGap plugin for using the mobile’s cambera is added by writing the following bash command in your project root folder:
\begin{verbatim}
$ cordova plugin add org.apache.cordova.camera
\end{verbatim}
The code for taking a picture with the camera in your mobile application is:
\begin{verbatim}
navigator.camera.getPicture(cameraSuccess, cameraError, cameraOptions);
\end{verbatim}
Where “cameraSucess” and “cameraError” is a callback function. 

To build and run the project on your mobile for testing you run in the root folder of your project:
\begin{verbatim}
$ cordova build android; 
$ cordova run android;
\end{verbatim}

For the debbuging the mobile application you can open desktop browser and where you can see the HTML, CSS and JavScript code and debugg the same way as a web application debugging is done in a web browser inspector. More can be read about debugging a PhoneGap application at \url{https://github.com/phonegap/phonegap/wiki/Debugging-in-PhoneGap}.



Thoughts:
 - The development of the architectures of the different methods can not be ensured to be entirely independent. Thus the results might vary depending on the order the development methods get evaluated. Also they're both reliant on the webapplication, if this one gets developed early on, they will have to conform to its standards, thus limiting the developers freedom. 

Positive / negative experiences
\begin{itemize}
    \item We are both skilled in java
    \item Very basic knowledge of webprogramming
    \item Our experience of developing in android
    \item Our experience of developing using phonegap
    \item Other
\end{itemize}

Diskussion (Discussion) möjliggör en längre diskussion och tolkning av resultaten från utvärderingsavsnittet, inklusive extrapoleringar och/eller förväntade resultat. Här passar det också att beskriva positiva och negativa erfarenheter relaterade till det arbete du har utfört.

\section{Recommendations}
%here we write a recommendation. If somebody would ask you you what to use to implement this specific application, you'd say what and why
\fi