\chapter{Introduction}\label{chapter-introduction}
The first section, Motivation \ref{section-motivation}, of this chapter gives a motivation for the research. Explaining why the research and in which context this is relevant. Thereafter the problem formulation \ref{section-problem-formulation} section is stated which describes the problem that is researched<<. In the Problem formulation section the scope of the study and the development goals are also stated. Development goals describes development goals that are important to make the research result applicable>>. The following section Contribution statement \ref{section-contribution-statement} contains the contribution to the research made by the researchers. The section also describes the knowledge gained by this research. Lastly the section Report organization \ref{section-report-organization} explains what the following chapters contain and how they fit together.

\section{Motivation}\label{section-motivation}
A web application has a lot of advantages compared to mobile applications. They run on all plattforms, have instant updates and are simple to maintain \cite{michaels2013}. However, a web application can't access some native mobile features, such as the mobile's accelorometer. A company who want to access such native features need to develop a mobile application. Developing a mobile application can be very expensive. Small to medium sized project for an Android mobile is estimated to cost from 20 000 - 40 000 dollars \cite{kohan2015}.

If a company has a web application they can choose to develop a mobile application from the ground up. The users activity in the mobile and web application can be synchronized with the use of the same back-end, see figure \ref{figure-common-backend}. Development time will be spent on: 
\begin{itemize}
\item Developing a new mobile application.
\item Maintaining the mobile and web application.
\item Restructuring the web application back-end to a common back-end for the mobile and web application.
\end{itemize}

% example exists at http://www.texample.net/tikz/examples/marketing-distribution-channel/
\begin{figure}\label{figure-common-backend}
\centering
\begin{tikzpicture}[node distance=1cm, auto]  
\tikzset{
    mynode/.style={rectangle,rounded corners,draw=black, top color=white, bottom color=yellow!50,very thick, inner sep=1em, minimum size=3em, text centered},
    myarrow/.style={->, >=latex', shorten >=1pt, thick},
    mylabel/.style={text width=7em, text centered} 
}  
\node[mynode] (backend) {Backend};  
\node[below=3cm of backend] (dummy) {}; 
\node[mynode, left=of dummy] (mobile) {Mobile application};  
\node[mynode, right=of dummy] (web) {Web application};

\draw[myarrow] (backend.south)  -- ++(0,-1) -|  (mobile.north);
\draw[myarrow] (mobile.north)  -- ++(0,1.506) -|  (backend.south);
\draw[myarrow] (backend.south)  -- ++(0,-1) -|  (web.north);
\draw[myarrow] (web.north)  -- ++(0,1.506) -|  (backend.south);
% There is a slight overlap of the arrows with the (manufacturer) south edge

\end{tikzpicture} 
\medskip
\caption{Seperate mobile and web application connected with a common backend.} 
\end{figure}

Another alternative, is to create a mobile application using the existing web application. The web application is run within a shell of the mobile application, see figure \ref{figure-encapsulated-web}. The web application can then access the mobile’s native functions. This means that the web application back-end does not need to be modified and instead of maintaining several platforms for every feature the mobile application only need to be maintained in regards of the use of the native functions. 

\begin{figure}\label{figure-encapsulated-web}
\centering
\begin{tikzpicture}[node distance=1cm, auto]  
\tikzset{
    mynode/.style={rectangle,rounded corners,draw=black, top color=white, bottom color=yellow!50,very thick, inner sep=1em, minimum size=3em, text centered},
    myarrow/.style={->, >=latex', shorten >=1pt, thick},
    mylabel/.style={text width=7em, text centered} 
}  
\node[mynode] (backend) {Backend};  
\node[below=3cm of backend] (dummy) {}; 
\node[mynode, left=of dummy] (mobile) {Mobile application};  
\node[mynode, right=of dummy] (web) {Web application};

\draw[myarrow] (backend.south)  -- ++(0,-1) -|  (web.north);
\draw[myarrow] (web.north)  -- ++(0,1.506) -|  (backend.south);
\draw[myarrow] (web.west)  -- (mobile.east);
\draw[myarrow] (mobile.east)  -- (web.west);
% There is a slight overlap of the arrows with the (manufacturer) south edge
% because creating the offset in another way didn't compile. 

\end{tikzpicture} 
\medskip
\caption{Seperate mobile and web application connected with a common backend.} 
\end{figure}

There are a number of researches of developing a mobile application using different methods. The researches focuses on aspects such as cost, effort and advantages and disadvantages for developing an entirely new mobile application. However, it is very hard to find research on extending an existing web application with mobile native features. 

The aim of this thesis is to evaluate two different development methods for encapsulating an existing web application in a mobile application to utilize native functions. With focus on evaluating development effort.

\section{Problem formulation}\label{section-problem-formulation}
The goal of this project is to evaluate two different development methods for encapsulating an existing web application in an Android mobile application. When evaluating the methods the focus is on development effort. The purpose of the mobile application is to give the web application access to the mobile's native functions. An example of such a native function would be accessing the mobile's accelerometer. It is important that the logic of the web application can be written in a general way (non-platform dependant) adapting to the device accessing the web application. So that the same web application code is used for the web application and mobile application.

Antoher goal is that the mobile application only provides data from native functions. In order to achieve that the web application and mobile application should have a master slave relationship. Where the web application acts as the master and the mobile application as the slave. To get data from the mobiles native functions the web application asks for the data and the mobile application passes the data back.

To enable this master and slave relationship the mobile and web application layer must be able to communicate. Passing commands and data between the layers. It is important the mobile and web application layers has a way of communicating that is developer friendly. 

To limit the scope of the research the development methods is only evaluted for the mobile operating system Android. Android was choosen since it is the most widely used today for mobiles.

There are two development methods of creating an Android application that will be evaluated. The first development method is to develop natively in Android. The second method is to develop using the framework PhoneGap.

The methods will be evaluated and compared from a developers perspective, with the preconditions defined section \ref{section-preconditions}.

\subsection{Preconditions} \label{section-preconditions}
The study is done from a company's perspective and is based on the following prerequisites:
\begin{itemize}
\item There are only a few developers, three or less.
\item There is already an existing web-application that is to be used by the mobile-application.
\item The company has a need and/or desire to extend the functionality of it's web-application with native functions.
\end{itemize}

\subsection{Scope} \label{subsection-scope}
%the scope of the research

\subsection{Development goals} \label{subsection-development-goals}
%the development goals

\section{Contribution statement}\label{section-contribution-statement}
Throughout this work we have been working together very closely. We have reviewed and improved each others work continuously. During development we often been pair programming. With that said we can attribute parts of our work to one of us more then the other. Even though most of the work has been created together. 

Philip designed the code of the web application and the mobile application built natively in Android. David designed the code for the PhoneGap application. David researched different techniques for the communication between the PhoneGap layer and the web application layer. David wrote the Approach section and Philip wrote the Introduction section in this paper. 

% here we also write what we achieved with this work

\section{Report organization}\label{section-report-organization}
%explain what the following chapters contain and how they fit together (one sentence each)